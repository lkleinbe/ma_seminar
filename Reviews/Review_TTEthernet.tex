%! Author = Leonard
%! Date = 23.06.2023
% Preamble
\documentclass[12pt,british,a4paper]{article}
% Packages
%! Author = Leonard
%! Date = 23.06.2023


% PACKAGES
%%%%%%%%%%%%%%%%%%%%%%%%%%%%%%%%%%%%%%%%%%%%%%%%%%%%%%%%%%%%%%%%%%%%%%%%%
\usepackage{titlesec}
\usepackage{amsmath}
\usepackage{amssymb}
\usepackage{stmaryrd}
\usepackage{tikz}
\usepackage{xfrac}
\usepackage{pdflscape}
\usepackage[official]{eurosym}
\usepackage{substr}
\usepackage{mathrsfs}
\usepackage{enumitem}
% Standard Packages
% -----------------------------------------------------------------------
\usepackage[british]{babel}
\usepackage{geometry} % alternatively type-area
\usepackage{fancyhdr}


% Font and Encoding Packages
% -----------------------------------------------------------------------
\usepackage[T1]{fontenc}    % enables symbols as ä, ö, ü, ß
\usepackage[utf8]{inputenc} % uft8-encoded symbols can processed
\usepackage{amsfonts}       % fonts mainly used in a mathematical context
\usepackage[lighttt]{lmodern}
%\usepackage[scaled=0.85]{DejaVuSansMono}

% Symbols
% -----------------------------------------------------------------------
% Specialized symbols
\usepackage{amssymb}	 % collection of mathematical symbols
\usepackage{extarrows}	 % collection of various arrows


% Maths Packages
% -----------------------------------------------------------------------
% These packages provide environments and commands for mathematical purposes
\usepackage{mathtools} % huge collection for mathematical writing replaces/extends amsmath
\usepackage{amsthm}    % provides claim-environments like theorem, lemma
\usepackage{cases}     % provides environment 'cases' for case distinctions
\usepackage{nicefrac}  % provides cmd 'nicefrac' to write '1/2' in texts


% Table and Picture Packages
% -----------------------------------------------------------------------
\usepackage{array}      % provides cmd to program talbes, e.g. with of a column
\usepackage{enumitem}   % user control over the basic list environments itemize, enumerate, and description
\usepackage{graphicx}   % provides optional arguments for 'includegraphics' like scaling
\usepackage{caption}    % enables description/captions for images, tables, algorithms
\usepackage{float}      % provides the 'H' option for floating objects like figures
\usepackage{pdfpages}   % enables the inclusion of PDFs in a latex document


% Computer Science Packages
% -----------------------------------------------------------------------
\usepackage[small]{complexity} % tool set for complexity theory
\usepackage{listings}          % integrate program source code into the pdf
\usepackage{algorithm}         % provides basic environment for algorithm display
\usepackage{algpseudocode}     % provides commands to write pseudocode


% Citation, Reference, Colours
% -----------------------------------------------------------------------
\usepackage{hyperref} % provides clickable in-document-references
\usepackage{xcolor}   % everything related to colours


\newcommand{\submissiontitle}[1]{
	\pagestyle{fancy}
	\setlength{\headheight}{1pt}
	\fancyhead[R]{\thepage}
	\fancyfoot{}
	\title{Review for "#1"}
	\author{}
	\date{}
}

% Document
\submissiontitle{Synthesis of Communication Schedules for
TTEthernet-Based Mixed-Criticality Systems}
\begin{document}
\maketitle
\section{Description}
In the Report the findings of the Paper "Synthesis of Communication Schedules for
TTEthernet-Based Mixed-Criticality Systems" are presented.
Additionally, the necessary backgrounds are presented.
Time Triggered Ethernet (TTE) is a communication protocol, which enables deterministic communication in networks with guarantees.
To achieve that, a static schedule is needed.
This schedule determines at which time a given node will forward a given frame.
The goal of the paper is to create such a schedule, which satisfies time constraints and minimises delay. To create the schedule, the heuristic TTESO strategy is used, which first creates a initial solution with List scheduling. This solution is then further improved with Tabu Search.
\section{Strengths}
\begin{itemize}
    \item The subject topic was presented in an understandable way.
    The examples were especially helpful in understanding the topic.
    \item The Background is presented without unnecessary information enabling readers to understand the topic without prior knowledge.
    \item Especially Figure 3 was very helpful to comprehend the analysis framework.
    I also like the annotation of parts of this figure to reference them in the text and explain the figure in detail.
    \item I personally found the usage of ChatGPT for image creation a good use-case and will certainly try this out.
\end{itemize}
\section{Weaknesses}\begin{itemize}
    \item In the Report a lot of abbreviations and acronyms are used.
    This is obviously caused by the need to analyse the subject in a formal matter and therefore can not be avoided. However using a lot of acronyms makes reading the paper hard for readers without a subject related background. I suggest adding a informal annex/section/table after the References to list all acronyms and their written out description.
    \item The order of the figures is a little confusing.
    One example is figure 2, which is only once referenced in Section 3.a but after Figure 6 and 3 have used already a few times. One possible solution is to rearrange the order. However think some references to figures are still missing. E.g. I assume figure 2 was to be referenced and introduced on Page 2 already to explain the concept of dataflow paths.
    \item At the end of Section III a reference is broken.
    \item There is no evaluation of the proposed algorithms.
    Usually the proposed algorithm is compared to other state-of-the-art algorithms or theoretical limits. This would also give you the ability to compare your simulation results to the algorithm in the paper. In the original paper they used a simulation benchmark for this as far as I can see. I suggest either presenting the performance of the algorithm in general or comparing your results to the one in the paper.
    \item I suggest not submitting the seminar work under the same title as the paper itself. Something like "Seminar Report for $\dots$ " distinguishes your work from the original work. You could also add this information to the abstract. However, I think that is optional if your title is expressive enough.
\end{itemize}
\end{document}