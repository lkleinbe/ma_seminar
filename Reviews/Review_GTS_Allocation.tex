%! Author = Leonard
%! Date = 23.06.2023
% Preamble
\documentclass[12pt,british,a4paper]{article}
% Packages
%! Author = Leonard
%! Date = 23.06.2023


% PACKAGES
%%%%%%%%%%%%%%%%%%%%%%%%%%%%%%%%%%%%%%%%%%%%%%%%%%%%%%%%%%%%%%%%%%%%%%%%%
\usepackage{titlesec}
\usepackage{amsmath}
\usepackage{amssymb}
\usepackage{stmaryrd}
\usepackage{tikz}
\usepackage{xfrac}
\usepackage{pdflscape}
\usepackage[official]{eurosym}
\usepackage{substr}
\usepackage{mathrsfs}
\usepackage{enumitem}
% Standard Packages
% -----------------------------------------------------------------------
\usepackage[british]{babel}
\usepackage{geometry} % alternatively type-area
\usepackage{fancyhdr}


% Font and Encoding Packages
% -----------------------------------------------------------------------
\usepackage[T1]{fontenc}    % enables symbols as ä, ö, ü, ß
\usepackage[utf8]{inputenc} % uft8-encoded symbols can processed
\usepackage{amsfonts}       % fonts mainly used in a mathematical context
\usepackage[lighttt]{lmodern}
%\usepackage[scaled=0.85]{DejaVuSansMono}

% Symbols
% -----------------------------------------------------------------------
% Specialized symbols
\usepackage{amssymb}	 % collection of mathematical symbols
\usepackage{extarrows}	 % collection of various arrows


% Maths Packages
% -----------------------------------------------------------------------
% These packages provide environments and commands for mathematical purposes
\usepackage{mathtools} % huge collection for mathematical writing replaces/extends amsmath
\usepackage{amsthm}    % provides claim-environments like theorem, lemma
\usepackage{cases}     % provides environment 'cases' for case distinctions
\usepackage{nicefrac}  % provides cmd 'nicefrac' to write '1/2' in texts


% Table and Picture Packages
% -----------------------------------------------------------------------
\usepackage{array}      % provides cmd to program talbes, e.g. with of a column
\usepackage{enumitem}   % user control over the basic list environments itemize, enumerate, and description
\usepackage{graphicx}   % provides optional arguments for 'includegraphics' like scaling
\usepackage{caption}    % enables description/captions for images, tables, algorithms
\usepackage{float}      % provides the 'H' option for floating objects like figures
\usepackage{pdfpages}   % enables the inclusion of PDFs in a latex document


% Computer Science Packages
% -----------------------------------------------------------------------
\usepackage[small]{complexity} % tool set for complexity theory
\usepackage{listings}          % integrate program source code into the pdf
\usepackage{algorithm}         % provides basic environment for algorithm display
\usepackage{algpseudocode}     % provides commands to write pseudocode


% Citation, Reference, Colours
% -----------------------------------------------------------------------
\usepackage{hyperref} % provides clickable in-document-references
\usepackage{xcolor}   % everything related to colours


\newcommand{\submissiontitle}[1]{
	\pagestyle{fancy}
	\setlength{\headheight}{1pt}
	\fancyhead[R]{\thepage}
	\fancyfoot{}
	\title{Review for "#1"}
	\author{}
	\date{}
}

% Document
\submissiontitle{Report About GTS Allocation Analysis in
IEEE 802.15.4 for Real-Time Wireless Sensor
Networks}
\begin{document}
\maketitle
\section{Description}
The Report is about the Paper "GTS Allocation Analysis in
IEEE 802.15.4 for Real-Time Wireless Sensor
Networks".
IEEE 802.15 is a networking protocol, which can operate in non-beacon-enabled mode and beacon enabled mode.
When beacon enabled mode is used, different devices are assigned guaranteed timeslots(GTS), in which they can send their data.
The Performance of the GTS is analysed in the paper.
To calculate the delay bound, network calculus is used to define service curve, arrival curve and ultimately the delay bound.
On the other hand normal calculus can be used to compute throughput and analyse the influence of frame size.
Finally, the throughput in different frequency bands is compared.
\section{Strengths}
\begin{itemize}
    \item The subject is presented in an understandable way.
    This is especially helpful, because the subject is very theoretical and might therefor not be intuitive for some readers.
    \item The necessary Background was presented without given too much detail.
    \item Especially Figure 2 was useful for understanding the system characteristics as well as the Arrival and Service curve definition.
\end{itemize}
\pagebreak
\section{Weaknesses}
I found your report of high quality and I have only very little things to nitpick.
\begin{itemize}
    \item I suggest, explaining IFS in more detail.
    The concept for IFS is very well explained in Section III but later formulas and even the Variable configuration use SIFS and LIFS.
    The connection between these parameters is not explained and might be confusing for readers without the necessary background.
    Further parameters like $N_{LIFS}$ are not explained at all which might make things even more confusing.
    \item When creating plots for your own report, some libraries allow to export into PGF, which can be directly imported into Latex.
    This allows Latex to render the plots on compile time, which increases sharpness and prevents unnecessary compression of the images.
    It also allows you to use a consistent font and even math symbols, like they are used in latex.
    E.g.\ when using matplotlib you can use something like this: \url{https://jwalton.info/Matplotlib-latex-PGF/}
    \item You should also interpret your plots drawing insights from your numerical results.
    On one Hand you can get direct insights how the GTS Performance behaves for different scenarios or interpret, how the performance behaves for different trends.#
    On the other Hand you can explicitly compare your results to the one in the paper.
    You did this for Fig 5 when you were explaining the results but your own figures are missing such an interpretation.
    \item I think there is only one author.
    The usage of the word "we" or "us" is not correct in that context.
\end{itemize}
\end{document}