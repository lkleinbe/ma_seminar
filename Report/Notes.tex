%! Author = Leonard
%! Date = 14.06.2023
% !TEX encoding = utf8
% !TeX spellcheck = en_gb

% Preamble
\documentclass[conference]{IEEEtran}
\IEEEoverridecommandlockouts
% The preceding line is only needed to identify funding in the first footnote. If that is unneeded, please comment it out.
% Packages
\usepackage{cite}
\usepackage{amsmath,amssymb,amsfonts}
\usepackage{hyperref}
\usepackage{algorithmic}
\usepackage{graphicx}
\usepackage{textcomp}
\usepackage{xcolor}
\usepackage{ieeetrantools}

% Document
\begin{document}
\title{On the Reliability of LTE Random Access: Performance Bounds for Machine-to-Machine Burst Resolution Time\\
{\footnotesize Seminar: Distributed Computer Systems SS23}}



\author{\IEEEauthorblockN{Leonard Kleinberger}
\IEEEauthorblockA{
    lkleinbe@rhrk.uni-kl.de}
}
\maketitle

\begin{abstract}
    tbd
\end{abstract}


\section{Introduction}
\begin{itemize}
    \item usecases:
    \begin{itemize}
        \item Power outage
        \item Fleet management
        \item Sensor networks
        \item Emergency situations/environmental desasters
    \end{itemize}
\end{itemize}
\section{Startup Procedure in LTE}
\begin{itemize}
    \item LTE System Setup
    \begin{itemize}
        \item UE, eNodeB, Core Network
        \item We ignore Protocol Stack mostly.
        We are only interested in Physical Layer to derive the necessary assumptions
        and from there on only look at message level
        \item We are interested in Setup until RRC is connected that is the connection setup between eNodeB and UE
        From there on the UE can contact the core via eNodeB but that is not in the scope of this work.
        RRC is will be disconnected when the UE is idle.
    \end{itemize}
    \item LTE Channels, Timing, Resource Grid
    \begin{itemize}
        \item Frames/Subframes/Symbols
        \item OFDM, multiple carriers
        \item Resource Grid
    \end{itemize}
    \item Over the Air Messages:
    \begin{itemize}
        \item Reference Signal, PSS, SSS
        \item MIB, SIB1, SIB2, more SIBs
        \item PRACH Preamble, RA Response, RRC Connection Request, RRC Connection Setup, RRC Connection Setup Complete
        \item until PRACH Preamble everything is only downlink -> no collisions
        \item PRACH Setup is only send in subframes specified by SIB2
        \item different Preambles do not collide.
        They use different sub-carrier.
        They also do not include any ue specific parameters.
        Number of available Preambles is specified in SIB2 and depends on cell usage and signal quality
        \item RA Response contains the grant in the Resource Grid
        \item During RRC Connection Request collision of multiple UEs.
        eNodeB might decode 1 message or nothing, we cant tell
        \item RRC Connection Setup will answer if message decoded successful; connected ue will realize; others will back of
        \item UE is successfully connected and responds with RRC Connection setup complete
    \end{itemize}
    \item Access Barring
    \begin{itemize}
        \item eNodeB can control how many ues try to connect
        \item cellBarred field in sib1 or sib2
        \item sib1 for all ues, sib2 for specific types of ues allowing emergency operators to work; Classes go from A to E.
        \item Static access barring can reduce the number of admitted ues significantly
        However statically setting p will slow down connection time when only a few ues are trying to connect
        \item dynamic access barring tries to adjust connect probability.
        \item optimal algorithm
        \item optimal algorithm is not applicable because eNodeB does not know how many UEs try to connect.
        An Estimation based on failed Preambles is needed.
        \item dynamic access barring with estimation
    \end{itemize}
\end{itemize}
\section{Analysis}
\begin{itemize}
    \item Assumptions
    \begin{itemize}
        \item We only look at number of PRACH contention periods, since the actual number of contention periods is up to the eNodeB
        \item RRC connections always fail, if a preamble is chosen by multiple ues.
        That is essentially the worst case
        \item we squash all 4 steps of RA Procedure into 1 Slot.
        That implies that every ue knows before the next start of the procedure, if the RA procedure before was a success
        ues dont content in multiple RA Procedures at the same time
        the eNodeB knows before the next RA Procedure how many ues successfully connected on last try
        \item we also assumes that the access barring probability can be updated between every timestep
        \item assumptions are pretty generous in my opinion.
        in reality the Ues only wait between 3-12 subframes before trying to PRACH again.
        also the change p every step implies we only have 1 contention per 2 frames, essentially limiting this analysis
        to the configurations with the lowest amount of prach phases.
    \end{itemize}
    \item QoS, System characteristics, $s_i$
    \begin{itemize}
        \item $SC = (M, N, access\_barring\_policy)$
        \item we use discrete timesteps to differentiate between the contention periods.
        \item at t = 0 N ues wake up
        \item $QOS = (b^\epsilon, t, \epsilon)$
        \item if a preamble is used multiple times is determined by:
        \[s_{i,m}= \begin{cases}
                       1& \text{if chosen by 1 UE}\\
                       0 &\text{otherwise}
        \end{cases}\]
        Note that this is essentially an M-channel slotted AlOHA system where each channel corresponds to a chosen preamble
    \end{itemize}
    \item Iterative Formula
    \begin{itemize}
        \item \[B(i+1) = \max\{0, B(i)+a_i-s_i\}\] with \[s_i = \sum_{m=0}^Ms_{i,m}\]
        \item this already allows analysis in a recursive/iterative way
        \item explanation
    \end{itemize}
    \item Probabilistic Analysis
    \begin{itemize}
        \item Formel 26
        \item first part can be computed fast using formula 25. Which uses formual 24 which uses formula 3
        \item second part is relatively slow using iterative formula. however $CM<<N$
        therefore it is faster than applying iterative formula for all
        \item explanation of MGF
    \end{itemize}
\end{itemize}
\section{Simulation}
\begin{itemize}
    \item Assumptions
    are pretty much the same as for the analysis.
    In the Paper the authors used Omnet++ which simulates the Physical Layer aswell
    For my implementation python is used.
    \item Implemented Iterative version as iterator
    \item Pseudocode
    \item using numpy for simulation to use c++ performance as far as possible.
    That includes throwing the dice for the access probability (binomial distribution), choosing the preambles (uniform)
    and counting how often each preamble was selected
    \item Running large number of simulations
    \item Multithreading over Searchspace\\
    Actually its multiprocessing because python interpreter lock which restrics the python interpreter to use only
    1 thread at the time to guarantee the consistency of shared objects.
\end{itemize}
\section{Simulation Results}
Simulations seem to support analysis.
Some configurations are not explicitly stated in the paper.
\section*{References}
\bibliographystyle{unsrturl}
\bibliography{literature}
%\bibliographystyle{IEEEtran}
%\bibliography{IEEEabrv,mybibfile}
\end{document}